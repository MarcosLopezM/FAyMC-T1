\documentclass[./../main.tex]{subfiles}
\graphicspath{{img/}}

\begin{document}
    \begin{exercise}
        Calcular el potencial de la barrera coulombiana \(V_{c}\) para las siguientes reacciones nucleares:

        \begin{itemize}
            \item \ch{^{12}C(p,p)^{12}}, \ch{^{12}C(d,p) ^{13}C}, \ch{^{12}C(d, ^{3}He) ^{11}B}, \ch{^{12}C(d, ^{4}He)^{10}B}
            \item \ch{^{27}Al(p,p) ^{27}Al}, \ch{^{27}Al(d,p) ^{28}Al}, \ch{^{27}Al(d, ^{3}He) ^{26}Mg}, \ch{^{27}Al(d, ^{4}He) ^{25}Mg}
            \item \ch{^{46}Fe(p,p) ^{46}Fe}, \ch{^{46}Fe(d,p) ^{47}Fe}, \ch{^{46}Fe(d, ^{3}He) ^{45}Mn}, \ch{^{46}Fe(d, ^{4}He)^{44}Mn}
            \item \ch{^{235}U(p,p) ^{235}U}
        \end{itemize}

        \begin{equation*}
            V_{c} = \dfrac{Z_{1}Z_{2}\e^{2}}{(R_{t} + r_{p})},
        \end{equation*}

        donde \(Z_{1}\) pertenece al blanco (\emph{target}), \(Z_{2}\) pertenece al proyectil, \(\e\) es la carga del electrón, \(R_{t}\) es el radio del blanco (\emph{target}) y \(r_{p}\) es el radio del proyectil.
    \end{exercise}
\end{document}