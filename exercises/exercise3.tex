\documentclass[./../main.tex]{subfiles}
\graphicspath{{img/}}

\begin{document}
    \color{blue}
    \begin{exercise}[Extra: Incertidumbre de \texorpdfstring{\(\observable{J}[x]\)}{Algo bien locote} y \texorpdfstring{\(\observable{J}[y]\)}{Algo bien locote también} (valor: +1pt)]
        Sea \(\observable{\hvect{J}}\) el operador general de momento angular con componentes \(\observable{J}[x]\), \(\observable{J}[y]\) y \(\observable{J}[z]\). En clase aprendimos que estas componentes no conmutan entre sí y por lo tanto deben satisfacer alguna relación de incertidumbre. Considerando la base de autoestados \(\ket{j, m}\) de los operadores \(\observable{J}[0][2]\) y \(\observable{J}[z]\), encuentra la relación de incertidumbre \(\change{\observable{J}[x]}\change{\observable{J}[y]}\).

        Recuerda que la definición de la desviación estándar de un operador \(\observable{\mathcal{O}}\) es

        \begin{equation*}
            \change{\observable{\mathcal{O}}} = \sqrt{\avg{\observable{\mathcal{O}}[][2]} - \avg{\observable{\mathcal{O}}}^{2}}.
        \end{equation*}

        \textbf{Sugerencia:} Utiliza los operadores de ascenso y descenso, \(\observable{J}[+] = \observable{J}[x] + i\observable{J}[y]\) y \(\observable{J}[-] = \observable{J}[x] - i\observable{J}[y]\), recordando que su acción sobre los estados \(\ket{j,m}\) es:

        \begin{equation*}
            \observable{J}[\pm]\ket{j,m} = \sqrt{j(j + 1) - m(m \pm 1)}\hbar\ket{j, m \pm 1}.
        \end{equation*}
    \end{exercise}
\end{document}