\documentclass[./../main.tex]{subfiles}
\graphicspath{{img/}}

\begin{document}
    \begin{exercise}[Perturbación en un sistema de tres estados (valor total: 5 pt)]
        Considera un sistema que solo tiene tres estados linealmente independientes. Considera ahora que el Hamiltoniano del sistema está dado por la siguiente matriz:

        \begin{equation*}
            \observable{H} = 
                V_{0}
                \matrice{1 - \epsilon, 0, 0, 0, 1, \epsilon, 0, \epsilon, 2},
        \end{equation*}

        en donde \(V_{0}\) es una constante y \(\epsilon\) es un número pequeño, es decir, \(\epsilon \ll 1\).

        \begin{enumerate}[label=(\alph*)]
            \item Valor: 0.5 pt - Si consideramos que \(\epsilon\) es ele resultado de una perturbación, escribe este Hamiltoniano como la suma de un Hamiltoniano imperturbado \(\observable{H}[0]\) y una perturbación \(\observable{W}\), es decir, \(\observable{H} = \observable{H}[0] + \observable{W}\), de tal forma que \(\observable{H} = \observable{H}[0]\) cuando \(\epsilon = 0\).
            \item Valor: 0.5 pt - ¿Quiénes son los eigenvalores y los eigenvectores del Hamiltoniano imperturbado \(\observable{H}[0]\)? Nota que este Hamiltoniano tiene un eigenvalor no degenerado y dos eigenvalores degenerados.
            \item Valor: 1.0pt - El Hamiltoniano \(\observable{H}\) puede resolverse exactamente Encuentra los eigenvalores exactos de \(\observable{H}\). Una vez que los hayas encontrado, exprésalos como una serie de potencias de Taylor en \(\epsilon\) hasta segundo orden, es decir, conserva todos los términos con orden igual o menor a \(\epsilon^{2}\).
            \item Valor: 1.0pt - Utiliza el caso \textbf{no} degenerado de la teoría de perturbaciones independientes del tiempo para calcular las correcciones a primer orden y segundo orden del eigenvalor \textbf{no} degenerado del Hamiltoniano imperturbado \(\observable{H}[0]\). Compara este resultado con el que encontraste en (c).
            \item Valor: 2.0pt - Utiliza ahora el caso degenerado de la teoría de perturbaciones independientes del tiempo para calcular las correcciones a primer orden de los dos eigenvalores degenerados de \(\observable{H}[0]\). Compara con los resultados del inciso (c).
        \end{enumerate}
    \end{exercise}
\end{document}