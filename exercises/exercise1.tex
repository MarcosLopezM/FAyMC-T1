% \PassOptionsToPackage{draft}{graphicx}
\documentclass[./../main.tex]{subfiles}

\begin{document}
    \begin{exercise}
        Calcular la masas total del planeta Tierra con los siguientes datos:

        Suponiendo que se trata de una esfera de radio \(r = \SI{6400}{\km}\); y que además la Tierra está constituida por los siguientes porcentajes de elementos: \SI{37}{\percent} de \ch{Fe}, \SI{25}{\percent} de \ch{Si}, \SI{13}{\percent} de \ch{Mg}, \SI{10}{\percent} de \ch{Ni}, \SI{8}{\percent} de \ch{Ca} y \SI{7}{\percent} de \ch{K}.

        Se deben consultarlas densidades de los componentes: \(\rho = \left[\unit[per-mode=fraction]{\density}\right]\) que se requieren para calcular la masa total de la Tierra.

        \begin{solution}
        Las densidades de los elementos se obtuvieron de \href{https://physics.nist.gov/cgi-bin/Star/compos.pl}{Material composition data}:

        \begin{alignat*}{3}
            \rho_{\ch{Fe}} &= \SI{7.87400}{\density},\quad \rho_{\ch{Si}} &{}={}& \SI{2.33000}{\density},\quad \rho_{\ch{Mg}} &{}={}& \SI{1.74000}{\density},\\
        \rho_{\ch{Ni}} &= \SI{8.90200}{\density},\quad \rho_{\ch{Ca}} &{}={}& \SI{1.55000}{\density},\qquad \rho_{\ch{K}} &{}={}& \SI{0.86200}{\density}.
        \end{alignat*}

        Y como queremos calcular la masa total de la Tierra, entonces la densidad total es:

        \begin{equation}
            \rho = 0.37 \rho_{\ch{Fe}} + 0.25 \rho_{\ch{Si}} + 0.13 \rho_{\ch{Mg}} + 0.10 \rho_{\ch{Ni}} + 0.08 \rho_{\ch{Ca}} + 0.07 \rho_{\ch{K}}.
            \label{eq:TotalDensity}
        \end{equation}

        Sustituyendo los valores de las densidades en \cref{eq:TotalDensity}

        \begin{align}
            \rho &= 0.37\left(\SI{7.87400}{\density}\right) + 0.25\left(\SI{2.33000}{\density}\right) + 0.13\left(\SI{1.74000}{\density}\right),\nonumber\\
            &+ 0.10\left(\SI{8.90200}{\density}\right) + 0.08\left(\SI{1.55000}{\density}\right) + 0.07\left(\SI{0.86200}{\density}\right),\nonumber\\
            \Acolorboxed{\rho &= \SI{4.79662}{\density}.}\label{eq:TotalDensityN}
        \end{align}

        Y, además, la masa total de la Tierra está dada por:

        \begin{align}
            M &= \rho V,\nonumber\\
            M &= \rho \left(\dfrac{4}{3}\pi r^{3}\right),\label{eq:Mass}
        \end{align}

        donde \(\frac{4}{3}\pi r^{3}\) es el volumen de una esfera.

        Sustituyendo \cref{eq:TotalDensityN} y el radio de la Tierra, \(r = \SI{6.4e8}{\cm}\), en \cref{eq:Mass}:

        \begin{align}
            M &= \SI{4.79662}{\density} \left(\dfrac{4}{3}\pi \left(\SI{6.4e8}{\cm}\right)^{3}\right),\nonumber\\
            &= \SI{5.26701e27}{\gram},\nonumber\\
            \Acolorboxed{M &\approx \SI{5.26701e24}{\kg}.}\label{eq:EarthMass}
        \end{align}
        \end{solution}
    \end{exercise}
\end{document}