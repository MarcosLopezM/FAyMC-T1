% \PassOptionsToPackage{draft}{graphicx}
\documentclass[./../main.tex]{subfiles}

\graphicspath{{img/}}

\begin{document}
    \begin{exercise}[Adición de momentos angulares: acoplamiento entre momento angular orbital y de spin (valor total: 5 pt)]
        Considera una partícula cuyos números cuánticos de momento angular orbital y de spin son \(j_{1} = 1\) y \(j_{2} = 1/2\), respectivamente. Realiza la adición de estos momentos angulares, es decir:

        \begin{enumerate}[label=(\alph*)]
            \item Valor: 1.0 pt - Determina los posibles valores de los números cuánticos \(j\) y \(m\) del sistema acoplado.
            \item Valor: 3.0 pt - Expresa los elementos de la base acoplada \(\set{\ket{j,m}}\) en términos de los elementos de la base desacoplada \(\set{\ket{j_{1}, j_{2}; m_{1}, m_{2}}}\). Para ello debes calcular ``a mano'' todos los coeficientes de Clebsch-Gordan involucrados, \(\braket{j_{1},j_{2}; m_{1}, m_{2}}{j,m}\).
            \item Valor 1.0 pt - Escribe la matriz de transformación entre estas dos bases.
        \end{enumerate}
    \end{exercise} 
\end{document}