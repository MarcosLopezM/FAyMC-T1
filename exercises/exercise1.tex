% \PassOptionsToPackage{draft}{graphicx}
\documentclass[./../main.tex]{subfiles}

\graphicspath{{img/}}

\begin{document}
    \begin{exercise}[Adición de momentos angulares: acoplamiento entre momento angular orbital y de spin (valor total: 5 pt)]
        Considera una partícula cuyos números cuánticos de momento angular orbital y de spin son \(j_{1} = 1\) y \(j_{2} = 1/2\), respectivamente. Realiza la adición de estos momentos angulares, es decir:

        \begin{enumerate}[label=(\alph*)]
            \item Valor: 1.0 pt - Determina los posibles valores de los números cuánticos \(j\) y \(m\) del sistema acoplado.
            
            \begin{solution}
                Recordemos que los posibles valores de \(j\) en la base acoplada vienen dados por:

                \begin{align*}
                    \abs{j_{1} - j_{2}} &\leq j \leq j_{1} + j_{2},\\
                    \dfrac{1}{2} &\leq j \leq \dfrac{3}{2}.
                \end{align*}

                Entonces los posibles valores de \(j\) son:

                \begin{empheq}[box = \color{pinkwave}\widefbox]{equation}
                    j = \dfrac{1}{2},\ \dfrac{3}{2}.\label{eq:j-values}
                \end{empheq}

                Por lo que los posibles valores de \(m\) son de la forma
                
                \begin{equation*}
                    -j \leq m \leq j,
                \end{equation*}

                tal que,                

                \begin{empheq}[box = \color{pinkwave}\widefbox]{equation}
                    \begin{alignedat}{2}
                        -\dfrac{1}{2} &\leq m &{}\leq{}& \dfrac{1}{2},\\
                        -\dfrac{3}{2} &\leq m &{}\leq{}& \dfrac{3}{2}.
                    \end{alignedat}
                    \label{eq:m-values}
                \end{empheq}
            \end{solution}
            
            \pagebreak
            \item Valor: 3.0 pt - Expresa los elementos de la base acoplada \(\set{\ket{j,m}}\) en términos de los elementos de la base desacoplada \(\set{\ket{j_{1}, j_{2}; m_{1}, m_{2}}}\). Para ello debes calcular ``a mano'' todos los coeficientes de Clebsch-Gordan involucrados, \(\braket{j_{1},j_{2}; m_{1}, m_{2}}{j,m}\).
            
            \begin{solution}
                Por \((2j_{1} + 1)(2j_{2} + 1)\) sabemos que la base acoplada debe tener 6 elementos y de \cref{eq:j-values,eq:m-values} tenemos que los elementos de la base acoplada (y para cada uno de lo subespacios) son:

                \begin{equation*}
                    \set{\ket{j,m}} = 
                    \begin{cases}
                        \ket{j = \frac{3}{2}, m} &\rightarrow \begin{cases}
                            \ket{\frac{3}{2}, \frac{3}{2}}\\
                            \ket{\frac{3}{2}, \frac{1}{2}}\\
                            \ket{\frac{3}{2}, -\frac{1}{2}}\\
                            \ket{\frac{3}{2}, -\frac{3}{2}}
                        \end{cases}\\
                        \ket{j = \frac{1}{2}, m} &\rightarrow \begin{cases}
                            \ket{\frac{1}{2}, \frac{1}{2}}\\
                            \ket{\frac{1}{2}, -\frac{1}{2}}
                        \end{cases}
                    \end{cases}
                \end{equation*}

                Recordemos ahora que los elementos de la base acoplada \(\set{\ket{j,m}}\) en términos de la base desacoplada \(\set{\ket{j_{1}, j_{2}; m_{1}, m_{2}}}\) están dados por:

                \begin{equation}
                    \ket{j, m} = \sum_{m_{1} = -j_{1}}^{j_{1}}\sum_{m_{2} = -j_{2}}^{j_{2}}\braket{j_{1}, j_{2}; m_{1}, m_{2}}{j, m}\ket{j_{1}, j_{2}; m_{1}, m_{2}}.
                    \label{eq:coupled-base-elements}
                \end{equation}

                Empezamos expresando el estado de máximo valor de \(j\) y \(m\), \(\ket{\frac{3}{2},\frac{3}{2}}\). Así,

                \begin{equation}
                    \ket{\tfrac{3}{2}, \tfrac{3}{2}} = \sum_{m_{1} = -1}^{1}\sum_{m_{2} = \tfrac{1}{2}}^{\tfrac{1}{2}}\braket{1, \tfrac{1}{2}; m_{1}, m_{2}}{\tfrac{3}{2}, \tfrac{3}{2}}\ket{1, \tfrac{1}{2}; m_{1}, m_{2}}.
                    \label{eq:j32-m32}
                \end{equation}

                Sin embargo, las reglas de relación de los CCG (Coeficientes de Clebsch-Gordan) nos exigen que

                \begin{equation*}
                    m = m_{1} + m_{2}.
                \end{equation*}

                Y para este caso tenemos que \(m = \frac{3}{2}\). Lo cual es válido únicamente cuando \(m_{1} = 1\) y \(m_{2} = \frac{1}{2}\), por lo que \cref{eq:j32-m32} se reduce a un solo término:

                \begin{equation*}
                    \ket{\tfrac{3}{2}, \tfrac{3}{2}} = \braket{1, \tfrac{1}{2}; 1, \tfrac{1}{2}}{\tfrac{3}{2}, \tfrac{3}{2}}\ket{1, \tfrac{1}{2}; 1, \tfrac{1}{2}}.
                \end{equation*}

                Además necesitamos la condición de normalización,

                \begin{align*}
                    \braket{1, \tfrac{1}{2}; 1, \tfrac{1}{2}}{\tfrac{3}{2}, \tfrac{3}{2}}^{2} &= 1,\\
                    \braket{1, \tfrac{1}{2}; 1, \tfrac{1}{2}}{\tfrac{3}{2}, \tfrac{3}{2}} &= \pm 1.
                \end{align*}

                Para determinar el signo del coeficiente usamos la condición de Condon-Shortley,

                \begin{equation*}
                    \braket{j_{1}, j_{2}; j_{1}, (j - j_{1})}{j, j} \geq 0.
                \end{equation*}

                Por lo que el signo de nuestro coeficiente es

                \begin{equation*}
                    \braket{1, \tfrac{1}{2}; 1, \tfrac{1}{2}}{\tfrac{3}{2}, \tfrac{3}{2}} = 1.
                \end{equation*}

                Entonces

                \begin{empheq}[box = \color{pinkwave}\widefbox]{equation}
                    \begin{alignedat}{1}
                        \ket{\tfrac{3}{2}, \tfrac{3}{2}} &= \ket{1, \tfrac{1}{2}; 1, \tfrac{1}{2}},\\
                        \braket{1, \tfrac{1}{2}; 1, \tfrac{1}{2}}{\tfrac{3}{2}, \tfrac{3}{2}} &= 1.\quad (\text{CCG})
                    \end{alignedat}
                    \label{eq:3232-CCG}
                \end{empheq}

                Ahora, para generar a \(\ket{\frac{3}{2}, \frac{1}{2}}\) podemos aplicar \(\observable{J}[-] = \observable{J}[1-] + \observable{J}[2-]\) a \cref{eq:3232-CCG},

                \begin{align}
                    \observable{J}[-]\ket{\tfrac{3}{2}, \tfrac{3}{2}} &= \observable{J}[1-]\ket{1, \tfrac{1}{2}; 1, \tfrac{1}{2}} + \observable{J}[2-]\ket{1, \tfrac{1}{2}; 1, \tfrac{1}{2}},\nonumber\\\
                    \hbar\sqrt{\tfrac{3}{2}\left(\tfrac{3}{2} + 1\right) - \tfrac{3}{2}\left(\tfrac{3}{2} - 1\right)}\ket{\tfrac{3}{2}, \tfrac{1}{2}} &= \begin{multlined}[t]
                        \hbar\sqrt{1(1 + 1) - 1(1 - 1)}\ket{1, \tfrac{1}{2}; 0, \tfrac{1}{2}}\\
                        + \hbar\sqrt{\tfrac{1}{2}\left(\tfrac{1}{2} + 1\right) - \tfrac{1}{2}\left(\tfrac{1}{2} - 1\right)}\ket{1, \tfrac{1}{2}; 1, -\tfrac{1}{2}},
                    \end{multlined}\nonumber\\
                    \Aboxedmain{\ket{\tfrac{3}{2}, \tfrac{1}{2}} &= \sqrt{\tfrac{2}{3}}\ket{1, \tfrac{1}{2}; 0, \tfrac{1}{2}} + \tfrac{1}{\sqrt{3}}\ket{1, \tfrac{1}{2}; 1, -\tfrac{1}{2}}.}\label{eq:3212-CCG}
                \end{align}

                Los CCG son

                \begin{empheq}[box = \color{customBlue} \fbox]{align*}
                    \braket{1, \tfrac{1}{2}; 0, \tfrac{1}{2}}{\tfrac{3}{2}, \tfrac{1}{2}} &= \sqrt{\tfrac{2}{3}},\\
                    \braket{1, \tfrac{1}{2}; 1, -\tfrac{1}{2}}{\tfrac{3}{2}, \tfrac{1}{2}} &= \tfrac{1}{\sqrt{3}}.
                \end{empheq}

                Repetimos el procedimiento: aplicamos \(\observable{J}[-]\) a \cref{eq:3212-CCG},

                \begin{align}
                    \observable{J}[-]\ket{\tfrac{3}{2}, \tfrac{1}{2}} &= \sqrt{\tfrac{2}{3}}(\observable{J}[1-] + \observable{J}[2-])\ket{1, \tfrac{1}{2}; 0, \tfrac{1}{2}} + \tfrac{2}{\sqrt{3}}(\observable{J}[1-] + \observable{J}[2-])\ket{1, \tfrac{1}{2}; 1, -\tfrac{1}{2}},\nonumber\\
                    2\ket{\tfrac{3}{2}, -\tfrac{1}{2}} &= \sqrt{\tfrac{2}{3}}\left[\sqrt{2}\ket{1, \tfrac{1}{2}; -1, \tfrac{1}{2}} + \ket{1, \tfrac{1}{2}; 0, -\tfrac{1}{2}}\right] + \tfrac{1}{\sqrt{3}}\left[\sqrt{2}\ket{1, \tfrac{1}{2}; 0, -\tfrac{1}{2}}\right],\nonumber\\
                    \Aboxedmain{\ket{\tfrac{3}{2}, -\tfrac{1}{2}} &= \tfrac{1}{\sqrt{3}}\ket{1, \tfrac{1}{2}; -1, \tfrac{1}{2}} + \sqrt{\tfrac{2}{3}}\ket{1, \tfrac{1}{2}; 0, -\tfrac{1}{2}}.}\label{eq:32-12-CCG}
                \end{align}

                \pagebreak
                Los CCG son:

                \begin{empheq}[box = \color{customBlue}\fbox]{align*}
                    \braket{\tfrac{3}{2}, -\tfrac{1}{2}}{\tfrac{3}{2}, -\tfrac{1}{2}} &= \tfrac{1}{\sqrt{3}},\\
                    \braket{1, \tfrac{1}{2}; 0, -\tfrac{1}{2}}{\tfrac{3}{2}, -\tfrac{1}{2}} &= \sqrt{\tfrac{2}{3}}.
                \end{empheq}

                Repetimos el procedimiento: aplicamos \(\observable{J}[-]\) a \cref{eq:32-12-CCG},

                \begin{align}
                    \observable{J}[-]\ket{\tfrac{3}{2}, -\tfrac{1}{2}} &= \tfrac{1}{\sqrt{3}}(\observable{J}[1-] + \observable{J}[2-])\ket{1, \tfrac{1}{2}; -1, \tfrac{1}{2}} + \sqrt{\tfrac{2}{3}}(\observable{J}[1-] + \observable{J}[2-])\ket{1, \tfrac{1}{2}; 0, -\tfrac{1}{2}},\nonumber\\
                    \Aboxedmain{\ket{\tfrac{3}{2}, -\tfrac{3}{2}} &= \ket{1, \tfrac{1}{2}; -1, -\tfrac{1}{2}}.}\label{eq:32-32-CCG}
                \end{align}

                El CCG es

                \begin{empheq}[box = \color{customBlue}\fbox]{equation*}
                    \braket{1, \tfrac{1}{2}; -1, -\tfrac{1}{2}}{\tfrac{3}{2}, -\tfrac{3}{2}}.
                \end{empheq}

                Ahora hay que encontrar \(\ket{\frac{1}{2}, \frac{1}{2}}\) que está en el subespacio \(\set{\ket{j = \frac{1}{2}, m}}\). Usamos \cref{eq:coupled-base-elements}:

                \begin{equation}
                    \ket{\tfrac{1}{2}, \tfrac{1}{2}} = \sum_{m_{1} = -1}^{1}\sum_{m_{2} = -\tfrac{1}{2}}^{\tfrac{1}{2}}\braket{1, \tfrac{1}{2}; m_{1}, m_{2}}{\tfrac{1}{2}, \tfrac{1}{2}}\ket{1, \tfrac{1}{2}; m_{1}, m_{2}}.
                    \label{eq:coupled-elements-2nd-subspace}
                \end{equation}

                Y sabemos que se deben cumplir las reglas de relación de los CCG, \(m = m_{1} + m_{2}\ \implies m = \frac{1}{2}\), lo cual se cumple únicamente cuando \(m_{1} = 0\) y \(m_{2} = \frac{1}{2}\), \(m_{1} = 1\) y \(m_{2} = -\frac{1}{2}\).

                \begin{equation*}
                    \ket{\tfrac{1}{2}, \tfrac{1}{2}} = \braket{1, \tfrac{1}{2}; 0, \tfrac{1}{2}}{\tfrac{1}{2}, \tfrac{1}{2}}\ket{1, \tfrac{1}{2}; 0, \tfrac{1}{2}} + \braket{1, \tfrac{1}{2}; 1, -\tfrac{1}{2}}{\tfrac{1}{2}, \tfrac{1}{2}}\ket{1, \tfrac{1}{2}; 1, -\tfrac{1}{2}}.
                \end{equation*}

                Definimos
                
                \begin{align*}
                    a &= \braket{1, \tfrac{1}{2}; 0, \tfrac{1}{2}}{\tfrac{1}{2}, \tfrac{1}{2}},\\
                    b &= \braket{1, \tfrac{1}{2}; 1, -\tfrac{1}{2}}{\tfrac{1}{2}, \tfrac{1}{2}}.
                \end{align*}

                Entonces,

                \begin{equation}
                    \ket{\tfrac{1}{2}, \tfrac{1}{2}} = a\ket{1, \tfrac{1}{2}; 0, \tfrac{1}{2}} + b\ket{1, \tfrac{1}{2}; 1, -\tfrac{1}{2}}.
                    \label{eq:1212-substitute-CCG}
                \end{equation}

                Y por la condición de normalización,

                \begin{equation}
                    a^{2} + b^{2} = 1.
                    \label{eq:normalization-condition-1212-CCG}
                \end{equation}

                \pagebreak
                Aplicamos \(\observable{J}[+]\) a \cref{eq:1212-substitute-CCG}

                \begin{align}
                    \observable{J}[+]\ket{\tfrac{1}{2}, \tfrac{1}{2}} &= a(\observable{J}[1+] + \observable{J}[2+])\ket{1, \tfrac{1}{2}; 0, \tfrac{1}{2}} + b(\observable{J}[1+] + \observable{J}[2+])\ket{1, \tfrac{1}{2}; 1, -\tfrac{1}{2}},\nonumber\\
                    0 &= \sqrt{2}a\ket{1, \tfrac{1}{2}; 1, \tfrac{1}{2}} + b\ket{1, \tfrac{1}{2}; 1, \tfrac{1}{2}},\nonumber\\
                    0 &= (\sqrt{2}a + b)\ket{1, \tfrac{1}{2}; 1, \tfrac{1}{2}},\nonumber\\
                    \sqrt{2}a + b &= 0.\label{eq:second-condition-1212-CCG}
                \end{align}

                Resolviendo simultáneamente \cref{eq:normalization-condition-1212-CCG,eq:second-condition-1212-CCG},

                \begin{empheq}[box = \fbox]{align*}
                    a &= \pm \tfrac{1}{\sqrt{3}},\\
                    b &= \mp \sqrt{\tfrac{2}{3}}.
                \end{empheq}
                
                Usamos la convención de Condon-Shortley, recordando la de \(a\) y que \(j = \frac{1}{2}\),
                
                \begin{empheq}[box = \fbox]{align*}
                    a &= -\tfrac{1}{\sqrt{3}},\\
                    b &= \sqrt{\tfrac{2}{3}}.
                \end{empheq}

                Por lo que \(\ket{\frac{1}{2}, \frac{1}{2}}\) queda como

                \begin{empheq}[box = \color{pinkwave}\widefbox]{equation}
                    \ket{\tfrac{1}{2}, \tfrac{1}{2}} = -\tfrac{1}{\sqrt{3}}\ket{1, \tfrac{1}{2}; 0, \tfrac{1}{2}} + \sqrt{\tfrac{2}{3}}\ket{1, \tfrac{1}{2}; 1, -\tfrac{1}{2}}.
                    \label{eq:1212-CCG}
                \end{empheq}

                Los CCG son:

                \begin{empheq}[box = \color{customBlue}\fbox]{align*}
                    \braket{1, \tfrac{1}{2}; 0, \tfrac{1}{2}}{\tfrac{1}{2}, \tfrac{1}{2}} &= -\tfrac{1}{\sqrt{3}},\\
                    \braket{1, \tfrac{1}{2}; 1, -\tfrac{1}{2}}{\tfrac{1}{2}, \tfrac{1}{2}} &= \sqrt{\tfrac{2}{3}}.
                \end{empheq}

                Aplicando \(\observable{J}[-1]\) a \cref{eq:1212-CCG}:

                \begin{align*}
                    \observable{J}[-]\ket{\tfrac{1}{2}, \tfrac{1}{2}} &=  -\tfrac{1}{\sqrt{3}}(\observable{J}[1-] + \observable{J}[2-])\ket{1, \tfrac{1}{2}; 0, \tfrac{1}{2}} + \sqrt{\tfrac{2}{3}}(\observable{J}[1-] + \observable{J}[2-])\ket{1, \tfrac{1}{2}; 1, -\tfrac{1}{2}},\\
                    \Aboxedmain{\ket{\tfrac{1}{2}, -\tfrac{1}{2}} &= -\sqrt{\tfrac{2}{3}}\ket{1, \tfrac{1}{2}; -1, \tfrac{1}{2}} + \tfrac{1}{\sqrt{3}}\ket{1, \tfrac{1}{2}; 0, -\tfrac{1}{2}}.}
                \end{align*}

                \pagebreak
                Los CCG son

                \begin{empheq}[box = \color{customBlue}\fbox]{align*}
                    \braket{1, \tfrac{1}{2}; -1, \tfrac{1}{2}}{\tfrac{1}{2}, -\tfrac{1}{2}} &= -\sqrt{\tfrac{2}{3}},\\
                    \braket{1, \tfrac{1}{2}; 0, -\tfrac{1}{2}}{\tfrac{1}{2}, -\tfrac{1}{2}} &= \tfrac{1}{\sqrt{3}}.
                \end{empheq}
            \end{solution}
            
            \item Valor 1.0 pt - Escribe la matriz de transformación entre estas dos bases.
            
            \begin{solution}
                La transformación puede escribirse como

                \begin{empheq}[box = \color{pinkwave}\widefbox]{equation*}
                    \matrice[1]{\ket{\frac{3}{2}, \frac{3}{2}},\ket{\frac{3}{2}, \frac{1}{2}},\ket{\frac{3}{2}, \frac{-1}{2}},\ket{\frac{3}{2}, \frac{-3}{2}},\ket{\frac{1}{2}, \frac{1}{2}},\ket{\frac{1}{2}, \frac{-1}{2}}} = 
                    \matrice{
                        1, 0, 0, 0, 0, 0,
                        0, \sqrt{\frac{2}{3}}, 0, \frac{1}{\sqrt{3}}, 0, 0,
                        0, 0, \frac{1}{\sqrt{3}}, 0, \sqrt{\frac{2}{3}}, 0,
                        0, 0, 0, 0, 0, 0,
                        0, \frac{-1}{\sqrt{3}}, 0, \sqrt{\frac{2}{3}}, 0, 0,
                        0, 0, -\sqrt{\frac{2}{3}}, \frac{1}{\sqrt{3}}, 0, 0
                        }
                    \matrice[1]{
                        \ket{1, \frac{1}{2}; 1, \frac{1}{2}},
                        \ket{1, \frac{1}{2}; 0, \frac{1}{2}},
                        \ket{1, \frac{1}{2}; -1, \frac{1}{2}},
                        \ket{1, \frac{1}{2}; 1, \frac{-1}{2}},
                        \ket{1, \frac{1}{2}; 0, \frac{-1}{2}},
                        \ket{1, \frac{1}{2}; -1, \frac{-1}{2}},
                    }
                \end{empheq}
            \end{solution}
        \end{enumerate}
    \end{exercise} 
\end{document}