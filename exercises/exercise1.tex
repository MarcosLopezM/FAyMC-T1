% \PassOptionsToPackage{draft}{graphicx}
\documentclass[./../main.tex]{subfiles}

\graphicspath{{img/}}

\begin{document}
    Un caso de reacción nuclear es cuando las partículas antes y después de una colisión son las mismas. De IBANDL se bajó la reacción \ch{^{12}C(p, p) ^{12} C}, medida a \SI{170}{\degree}. La primera columna del archivo anexo es la energía de bombardeo en unidades de \unit{\keV}, la segunda columna es la sección de dispersión elástica en unidades de \unit{\mb}.

    \begin{exercise}
        Calcular la barrera coulombiana \(V_{c}\) para esta reacción.
    \end{exercise}
\end{document}