% \PassOptionsToPackage{draft}{graphicx}
\documentclass[./../main.tex]{subfiles}

\graphicspath{{img/}}

\begin{document}
    \begin{exercise}
        Calcula el factor relativista \(\gamma\) de un protón de \qty{10}{\GeV} de energía total y de un electrón de \qty{1}{\GeV}.

        \begin{solution}
            Recordemos que el factor relativista \(\gamma\) está dado por:

            \begin{equation}
                \gamma = \dfrac{E}{m c^{2}}.
                \label{eq:relativistic-factor}
            \end{equation}

            Además, que las masas del electrón (\(m_{\e}\)) y protón (\(m_{p}\)) son \qty{0.511}{\MeV\per\clightsq} y \qty{938}{\MeV\per\clightsq}, respectivamente. Por lo tanto, el factor relativista del protón es:

            \begin{align*}
                \gamma_{p} &= \dfrac{\qty{10e9}{\eV}}{(\qty{938e6}{\eV\per\clightsq}) \cdot c^{2}},\\
                &= \dfrac{\qty{10e9}{\eV}}{\qty{938e6}{\eV}},\\
                \Acolorboxed{\gamma_{p} &= \qty{10.661}{\eV}.}
            \end{align*}

            Y para el electrón,

            \begin{align*}
                \gamma_{\e} &= \dfrac{\qty{1e9}{\eV}}{(\qty{0.511e6}{\eV\per\clightsq}) \cdot c^{2}},\\
                &= \dfrac{\qty{1e9}{\eV}}{\qty{0.511e6}{\eV}},\\
                \Acolorboxed{\gamma_{\e} &= \qty{1956.95}{\eV}.}
            \end{align*}
        \end{solution}
    \end{exercise}
\end{document}